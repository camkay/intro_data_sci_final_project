\documentclass[man]{apa6}
\usepackage{lmodern}
\usepackage{amssymb,amsmath}
\usepackage{ifxetex,ifluatex}
\usepackage{fixltx2e} % provides \textsubscript
\ifnum 0\ifxetex 1\fi\ifluatex 1\fi=0 % if pdftex
  \usepackage[T1]{fontenc}
  \usepackage[utf8]{inputenc}
\else % if luatex or xelatex
  \ifxetex
    \usepackage{mathspec}
  \else
    \usepackage{fontspec}
  \fi
  \defaultfontfeatures{Ligatures=TeX,Scale=MatchLowercase}
\fi
% use upquote if available, for straight quotes in verbatim environments
\IfFileExists{upquote.sty}{\usepackage{upquote}}{}
% use microtype if available
\IfFileExists{microtype.sty}{%
\usepackage{microtype}
\UseMicrotypeSet[protrusion]{basicmath} % disable protrusion for tt fonts
}{}
\usepackage{hyperref}
\hypersetup{unicode=true,
            pdftitle={Something Clever\ldots{}},
            pdfauthor={Cameron S. Kay, Stefania R. Ashby, \& Ashley L. Miller},
            pdfkeywords={american, social media, internet, pew research center},
            pdfborder={0 0 0},
            breaklinks=true}
\urlstyle{same}  % don't use monospace font for urls
\usepackage{graphicx,grffile}
\makeatletter
\def\maxwidth{\ifdim\Gin@nat@width>\linewidth\linewidth\else\Gin@nat@width\fi}
\def\maxheight{\ifdim\Gin@nat@height>\textheight\textheight\else\Gin@nat@height\fi}
\makeatother
% Scale images if necessary, so that they will not overflow the page
% margins by default, and it is still possible to overwrite the defaults
% using explicit options in \includegraphics[width, height, ...]{}
\setkeys{Gin}{width=\maxwidth,height=\maxheight,keepaspectratio}
\IfFileExists{parskip.sty}{%
\usepackage{parskip}
}{% else
\setlength{\parindent}{0pt}
\setlength{\parskip}{6pt plus 2pt minus 1pt}
}
\setlength{\emergencystretch}{3em}  % prevent overfull lines
\providecommand{\tightlist}{%
  \setlength{\itemsep}{0pt}\setlength{\parskip}{0pt}}
\setcounter{secnumdepth}{0}
% Redefines (sub)paragraphs to behave more like sections
\ifx\paragraph\undefined\else
\let\oldparagraph\paragraph
\renewcommand{\paragraph}[1]{\oldparagraph{#1}\mbox{}}
\fi
\ifx\subparagraph\undefined\else
\let\oldsubparagraph\subparagraph
\renewcommand{\subparagraph}[1]{\oldsubparagraph{#1}\mbox{}}
\fi

%%% Use protect on footnotes to avoid problems with footnotes in titles
\let\rmarkdownfootnote\footnote%
\def\footnote{\protect\rmarkdownfootnote}


  \title{Something Clever\ldots{}}
    \author{Cameron S. Kay\textsuperscript{1}, Stefania R. Ashby\textsuperscript{1},
\& Ashley L. Miller\textsuperscript{1}}
    \date{}
  
\shorttitle{Title}
\affiliation{
\vspace{0.5cm}
\textsuperscript{1} University of Oregon}
\keywords{american, social media, internet, pew research center}
\usepackage{csquotes}
\usepackage{upgreek}
\captionsetup{font=singlespacing,justification=justified}

\usepackage{longtable}
\usepackage{lscape}
\usepackage{multirow}
\usepackage{tabularx}
\usepackage[flushleft]{threeparttable}
\usepackage{threeparttablex}

\newenvironment{lltable}{\begin{landscape}\begin{center}\begin{ThreePartTable}}{\end{ThreePartTable}\end{center}\end{landscape}}

\makeatletter
\newcommand\LastLTentrywidth{1em}
\newlength\longtablewidth
\setlength{\longtablewidth}{1in}
\newcommand{\getlongtablewidth}{\begingroup \ifcsname LT@\roman{LT@tables}\endcsname \global\longtablewidth=0pt \renewcommand{\LT@entry}[2]{\global\advance\longtablewidth by ##2\relax\gdef\LastLTentrywidth{##2}}\@nameuse{LT@\roman{LT@tables}} \fi \endgroup}


\DeclareDelayedFloatFlavor{ThreePartTable}{table}
\DeclareDelayedFloatFlavor{lltable}{table}
\DeclareDelayedFloatFlavor*{longtable}{table}
\makeatletter
\renewcommand{\efloat@iwrite}[1]{\immediate\expandafter\protected@write\csname efloat@post#1\endcsname{}}
\makeatother

\authornote{

Correspondence concerning this article should be addressed to Cameron S.
Kay, 1451 Onyx Street, Eugene, OR 97403. E-mail:
\href{mailto:ckay@uoregon.edu}{\nolinkurl{ckay@uoregon.edu}}}

\abstract{
TBD


}

\usepackage{amsthm}
\newtheorem{theorem}{Theorem}[section]
\newtheorem{lemma}{Lemma}[section]
\theoremstyle{definition}
\newtheorem{definition}{Definition}[section]
\newtheorem{corollary}{Corollary}[section]
\newtheorem{proposition}{Proposition}[section]
\theoremstyle{definition}
\newtheorem{example}{Example}[section]
\theoremstyle{definition}
\newtheorem{exercise}{Exercise}[section]
\theoremstyle{remark}
\newtheorem*{remark}{Remark}
\newtheorem*{solution}{Solution}
\begin{document}
\maketitle

A common definition for a social media site is that it is an
internet-based service allowing for the creation and broadcast of
user-generated information (Boyd \& Ellison, 2008; Kaplan \& Haenlein,
2010; Obar \& Wildman, 2015). Obar and Wildman (2015) emphasize the
user-generated aspect of this definition, arguing that this content is
the lifeblood of social media. Although that may sound hyperbolic, it
logically follows that if a site is created with the express purpose of
providing user-generated content, it must have user-generated content to
function as intended. By way of illustration, without videos created by
users, YouTube, a social media site that allows its users to upload and
share videos, would fail to serve its primary purpose. Netflix, a site
that allows users to only stream videos, does not require user-generated
content, as it does not serve user-generated content, and, by extension
is not a social media site. Beyond the functional aspects of social
media sites, the user-generated focus also highlights the importance of
individual differences in the user-service relationship, as users
invariably have characteristics that affect how they consume and
generate content.

\section{Methods}\label{methods}

The data for the current study was collected by the Pew Research Center
(2018).

\subsection{Participants}\label{participants}

Two thousand, two people were surveyed by telephone (75.02\% cell phone;
24.98\% landline) over a period of 7 days in January of 2018. We
excluded any participants who reported that they do not even
occasionally use the internet or email (\emph{n} = 273). The resulting
sample comprised 1729 people (45.29\% female). Ages ranged from 18 to 97
(\emph{M} age = 48.29; \emph{SD} age =
17.94)\footnote{Note that the descriptive statistics for age are slightly lower than reality. Ages 97 and older were recorded as simply 97 in the data.}.
Concerning race, 68.48\% identified as white, 12.78\% identified as
black, 3.64\% identified as Asian, 2.95\% identified as mixed race, and
12.15\% refused to answer or reported being from some other race.

\subsection{Material}\label{material}

\subsection{Procedure}\label{procedure}

\subsection{Data analysis}\label{data-analysis}

We used R (Version 3.5.1; R Core Team, 2018) and the R-packages
\emph{bindrcpp} (Version 0.2.2; Müller, 2018), \emph{cowplot} (Version
0.9.3; Wilke, 2018), \emph{dplyr} (Version 0.7.8; Wickham, François,
Henry, \& Müller, 2018), \emph{forcats} (Version 0.3.0; Wickham, 2018a),
\emph{Formula} (Version 1.2.3; Zeileis \& Croissant, 2010),
\emph{ggplot2} (Version 3.1.0; Wickham, 2016), \emph{here} (Version 0.1;
Müller, 2017), \emph{Hmisc} (Version 4.1.1; Harrell Jr, Charles Dupont,
\& others., 2018), \emph{lattice} (Version 0.20.38; Sarkar, 2008),
\emph{lme4} (Version 1.1.19; Bates, Mächler, Bolker, \& Walker, 2015),
\emph{lmerTest} (Version 3.0.1; Kuznetsova, Brockhoff, \& Christensen,
2017), \emph{lubridate} (Version 1.7.4; Grolemund \& Wickham, 2011),
\emph{magrittr} (Version 1.5; Bache \& Wickham, 2014), \emph{Matrix}
(Version 1.2.15; Bates \& Maechler, 2018), \emph{pander} (Version 0.6.3;
Daróczi \& Tsegelskyi, 2018), \emph{papaja} (Version 0.1.0.9842; Aust \&
Barth, 2018), \emph{purrr} (Version 0.2.5; Henry \& Wickham, 2018),
\emph{readr} (Version 1.1.1; Wickham, Hester, \& Francois, 2017),
\emph{rio} (Version 0.5.10; C.-h. Chan, Chan, Leeper, \& Becker, 2018),
\emph{stringr} (Version 1.3.1; Wickham, 2018b), \emph{survival} (Version
2.43.1; Terry M. Therneau \& Patricia M. Grambsch, 2000), \emph{tibble}
(Version 1.4.2; Müller \& Wickham, 2018), \emph{tidyr} (Version 0.8.2;
Wickham \& Henry, 2018), \emph{tidyverse} (Version 1.2.1; Wickham,
2017), and \emph{wesanderson} (Version 0.3.6; Ram \& Wickham, 2018) for
all our analyses.

\section{Results}\label{results}

\section{Discussion}\label{discussion}

\newpage

\section{References}\label{references}

\begingroup
\setlength{\parindent}{-0.5in} \setlength{\leftskip}{0.5in}

\hypertarget{refs}{}
\hypertarget{ref-R-papaja}{}
Aust, F., \& Barth, M. (2018). \emph{papaja: Create APA manuscripts with
R Markdown}. Retrieved from \url{https://github.com/crsh/papaja}

\hypertarget{ref-R-magrittr}{}
Bache, S. M., \& Wickham, H. (2014). \emph{Magrittr: A forward-pipe
operator for r}. Retrieved from
\url{https://CRAN.R-project.org/package=magrittr}

\hypertarget{ref-R-Matrix}{}
Bates, D., \& Maechler, M. (2018). \emph{Matrix: Sparse and dense matrix
classes and methods}. Retrieved from
\url{https://CRAN.R-project.org/package=Matrix}

\hypertarget{ref-R-lme4}{}
Bates, D., Mächler, M., Bolker, B., \& Walker, S. (2015). Fitting linear
mixed-effects models using lme4. \emph{Journal of Statistical Software},
\emph{67}(1), 1--48.
doi:\href{https://doi.org/10.18637/jss.v067.i01}{10.18637/jss.v067.i01}

\hypertarget{ref-Boyd2008}{}
Boyd, D. M., \& Ellison, N. B. (2008). Social networking sites:
Definitions, history, and scholarship. \emph{Journal of
Computer-Mediated Communication}, \emph{13}, 210--230.

\hypertarget{ref-R-rio}{}
Chan, C.-h., Chan, G. C., Leeper, T. J., \& Becker, J. (2018).
\emph{Rio: A swiss-army knife for data file i/o}.

\hypertarget{ref-R-pander}{}
Daróczi, G., \& Tsegelskyi, R. (2018). \emph{Pander: An r 'pandoc'
writer}. Retrieved from \url{https://CRAN.R-project.org/package=pander}

\hypertarget{ref-R-lubridate}{}
Grolemund, G., \& Wickham, H. (2011). Dates and times made easy with
lubridate. \emph{Journal of Statistical Software}, \emph{40}(3), 1--25.
Retrieved from \url{http://www.jstatsoft.org/v40/i03/}

\hypertarget{ref-R-Hmisc}{}
Harrell Jr, F. E., Charles Dupont, \& others. (2018). \emph{Hmisc:
Harrell miscellaneous}. Retrieved from
\url{https://CRAN.R-project.org/package=Hmisc}

\hypertarget{ref-R-purrr}{}
Henry, L., \& Wickham, H. (2018). \emph{Purrr: Functional programming
tools}. Retrieved from \url{https://CRAN.R-project.org/package=purrr}

\hypertarget{ref-Kaplan2010}{}
Kaplan, A. M., \& Haenlein, M. (2010). Users of the world, unite! The
challenges and opportunities of Social Media. \emph{Business Horizons},
\emph{53}, 59--68.
doi:\href{https://doi.org/10.1016/j.bushor.2009.09.003}{10.1016/j.bushor.2009.09.003}

\hypertarget{ref-R-lmerTest}{}
Kuznetsova, A., Brockhoff, P. B., \& Christensen, R. H. B. (2017).
lmerTest package: Tests in linear mixed effects models. \emph{Journal of
Statistical Software}, \emph{82}(13), 1--26.
doi:\href{https://doi.org/10.18637/jss.v082.i13}{10.18637/jss.v082.i13}

\hypertarget{ref-R-here}{}
Müller, K. (2017). \emph{Here: A simpler way to find your files}.
Retrieved from \url{https://CRAN.R-project.org/package=here}

\hypertarget{ref-R-bindrcpp}{}
Müller, K. (2018). \emph{Bindrcpp: An 'rcpp' interface to active
bindings}. Retrieved from
\url{https://CRAN.R-project.org/package=bindrcpp}

\hypertarget{ref-R-tibble}{}
Müller, K., \& Wickham, H. (2018). \emph{Tibble: Simple data frames}.
Retrieved from \url{https://CRAN.R-project.org/package=tibble}

\hypertarget{ref-Obar2015}{}
Obar, J. A., \& Wildman, S. (2015). Social media definition and the
governance challenge: An introduction to the special issue.
\emph{Telecommunications Policy}, \emph{39}(9), 745--750.

\hypertarget{ref-Pew}{}
Pew Research Center. (2018). Core trends survey. Retrieved from
\url{http://www.pewinternet.org/dataset/jan-3-10-2018-core-trends-survey/}

\hypertarget{ref-R-base}{}
R Core Team. (2018). \emph{R: A language and environment for statistical
computing}. Vienna, Austria: R Foundation for Statistical Computing.
Retrieved from \url{https://www.R-project.org/}

\hypertarget{ref-R-wesanderson}{}
Ram, K., \& Wickham, H. (2018). \emph{Wesanderson: A wes anderson
palette generator}. Retrieved from
\url{https://CRAN.R-project.org/package=wesanderson}

\hypertarget{ref-R-lattice}{}
Sarkar, D. (2008). \emph{Lattice: Multivariate data visualization with
r}. New York: Springer. Retrieved from
\url{http://lmdvr.r-forge.r-project.org}

\hypertarget{ref-R-survival-book}{}
Terry M. Therneau, \& Patricia M. Grambsch. (2000). \emph{Modeling
survival data: Extending the Cox model}. New York: Springer.

\hypertarget{ref-R-ggplot2}{}
Wickham, H. (2016). \emph{Ggplot2: Elegant graphics for data analysis}.
Springer-Verlag New York. Retrieved from \url{http://ggplot2.org}

\hypertarget{ref-R-tidyverse}{}
Wickham, H. (2017). \emph{Tidyverse: Easily install and load the
'tidyverse'}. Retrieved from
\url{https://CRAN.R-project.org/package=tidyverse}

\hypertarget{ref-R-forcats}{}
Wickham, H. (2018a). \emph{Forcats: Tools for working with categorical
variables (factors)}. Retrieved from
\url{https://CRAN.R-project.org/package=forcats}

\hypertarget{ref-R-stringr}{}
Wickham, H. (2018b). \emph{Stringr: Simple, consistent wrappers for
common string operations}. Retrieved from
\url{https://CRAN.R-project.org/package=stringr}

\hypertarget{ref-R-tidyr}{}
Wickham, H., \& Henry, L. (2018). \emph{Tidyr: Easily tidy data with
'spread()' and 'gather()' functions}. Retrieved from
\url{https://CRAN.R-project.org/package=tidyr}

\hypertarget{ref-R-dplyr}{}
Wickham, H., François, R., Henry, L., \& Müller, K. (2018). \emph{Dplyr:
A grammar of data manipulation}. Retrieved from
\url{https://CRAN.R-project.org/package=dplyr}

\hypertarget{ref-R-readr}{}
Wickham, H., Hester, J., \& Francois, R. (2017). \emph{Readr: Read
rectangular text data}. Retrieved from
\url{https://CRAN.R-project.org/package=readr}

\hypertarget{ref-R-cowplot}{}
Wilke, C. O. (2018). \emph{Cowplot: Streamlined plot theme and plot
annotations for 'ggplot2'}. Retrieved from
\url{https://CRAN.R-project.org/package=cowplot}

\hypertarget{ref-R-Formula}{}
Zeileis, A., \& Croissant, Y. (2010). Extended model formulas in R:
Multiple parts and multiple responses. \emph{Journal of Statistical
Software}, \emph{34}(1), 1--13.
doi:\href{https://doi.org/10.18637/jss.v034.i01}{10.18637/jss.v034.i01}

\endgroup


\end{document}
